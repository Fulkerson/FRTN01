%\documentclass{tufte-handout}
\documentclass{article}
%\usepackage{amsmath,amsthm}
\usepackage{fontspec}
%\usepackage{unicode-math}
%\setmathfont{Palatino Linotype}
%\setmainfont{Palatino Linotype}
%\usepackage{minted}
%\usemintedstyle{bw}
\usepackage{hyperref}

\input{vc.tex}

%\newtheorem{claim}{Claim}[section]
\title{Control of Batch Tank using Raspberry Pi \\ FRTN01 project - Group 20}
\date{\GITDate,\\\small\GITAbrHash}
\author{Johan Anderholm \\ Jonathan Kämpe \\ Mikael Sahlström \\ Mikael Nilsson}

\begin{document}
\maketitle
\newpage
\tableofcontents
\newpage
\section{Problem description}
% An introduction that states the problem that has been solved.

\section{Program structure}
% A section describing the main program structure, both from a class
% and and a real-time perspective. If possible illustrate this with some
% type of figure.

\section{Control design}
% A section describing the control design aspects of the project.
\begin{description}
\item[$m$] Building of color matrices
\item[$2^n$] For all color combinations
\item[$n^3$] Determinant calculation
\item[$n^2$] Matrix addition/subtraction
\end{description}


\section{Running the system}
% A section describing the user interface in the project including a short
% HowTo description on how to start and operate the program.

\section{Results}
% A section containing the results. In case the project is a control-oriented
% project this should include plots of measurement signals, reference sig-
% nal, and control signal. If the project is more of a real-time nature then
% this section could contain measurement results of different type.

\section{Conclusion}
% A conclusion section.

Since the degree of the polynomial is at maximum n, by using Claim~2.1 in \cite[p.~2-1]{alistair} we get the bounded probability of less than $\frac{n}{p}$ where p is a previously fixed prime number.

\newpage

\begin{thebibliography}{9}
\bibitem{alistair}
\url{http://www.cs.berkeley.edu/~sinclair/cs271/n2.pdf}
\bibitem{testset}
\url{https://piazza.com/class#fall2012/edan55/50}
\end{thebibliography}

\end{document}
